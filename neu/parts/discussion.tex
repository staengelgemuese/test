\chapter{Conclusion}

The conclusion of this work consists of two parts. In the first part, we will summarise the results that we obtained in the previous chapters, followed by an outlook for future works in the second part. 

\section{Summary}
In this thesis, we have given an overview of all important equations and methods that have been used for the simulation using the \gls{cns} solver of \gls{bosss}. We have gained a fundamental knowledge about the \gls{dg} method which basically is a combination of \gls{fem} and \gls{fvm}. We have also studied the time marching algorithm that is used by \gls{bosss}, the \gls{rk} method and have especially been concerned with the \gls{cfl}-number for the explicit Euler time discretisation which is a main parameter for influencing the time step size of the simulations. Furthermore, we have dealt with the \gls{ibm}, which creates the mesh used in our simulations by overlaying a simple, structured mesh with a level set that defines the boundaries of the domain. \\\indent
After explaining the necessary theory, we have turned to the verification and validation of the \gls{cns} solver for inviscid and viscid flows. In \cref{eulerVerification}, we have proven the convergence rate of $\mathcal{O}(h^{P+1})$ for the simulation of Euler equations. We have also studied the robustness of the solver, which is sufficient for low- and moderate-order simulations, but improvable for high-order simulations.\\\indent
In the last chapter of this thesis, we have compared the results for lift and drag coefficient, wake separation length and Strouhal number for simulations of different Reynolds numbers, namely 20, 40, 100 and 200, for different mesh and polynomial degree properties. We have noted that at a similar number of \gls{dof}s the simulation of higher order yields better results. Furthermore, we have compared those results for our finest simulation to those of other studies and noted a very good agreement for all values. We can therefore come to the conclusion that the \gls{cns} solver using immersed boundaries is sufficiently validated for the simulation of compressible viscid flows. 

\section{Outlook}
For future works, we have observed two main issues. During the simulation and comparison of viscid flows in \cref{viscousCylinder}, we have noted a discordant value for DG2CpD80, which has always been higher than the expected result. This issue is slightly unsettling as we did not find a logical explanation and it should therefore be subject of future works. \\\indent
The other issue concerns the robustness of the \gls{cns} solver using immersed boundaries and is noticeable throughout the simulations. As we have observed during the robustness study in \cref{eulerVerification}, the simulations are often unstable for higher orders on coarse meshes. As the \gls{rkdg} method aims for the ability of obtaining proper results while using coarse and simple meshes, the use of high polynomial degrees is necessary. Unfortunately, we have not been able to obtain results for our coarsest mesh with the highest degree (DG3CpD40) in our viscous simulations in \cref{viscousCylinder}, because it has been necessary to use a time step size that small that the simulation time would have been tremendous. Therefore, future works should be devoted to this problem in order to make \gls{bosss} usable for industrial purposes. 

