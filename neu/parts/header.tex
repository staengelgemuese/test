	
	% % % % % % % % % % % % % % % % % % % % % % % % 
	%													  %
	% 	Definition der Dokumentenklasse und ihrer Einstellungen		%
	%													  %
	% % % % % % % % % % % % % % % % % % % % % % % %
	
	% Die verwendete TUD-Design-Klasse tudreport setzt auf der KOMA-Script-Klasse scrreprt auf und implementiert die Vorgaben des Corporate Design Handbuchs der TU Darmstadt. 
	% scrreprt bringt einige Voreinstellungen mit sich: 
	% 11pt, a4paper, abstractoff, bigheadings, final, footnosepline, headnosepline, nochapterprefix, onecolumn, oneside, openany, parindent, tablecaptionbelow und titlepage. 
	% Details zu diesen Voreinstellungen können der Dokumentation des KOMA-Scripts entnommen werden. Im folgenden werden lediglich Optionen definiert, welche von den Voreinstellungen abweichen oder durch die Klasse tudreport definiert werden.
	
	%\documentclass[accentcolor=tud7a, bigchapter, colorbacktitle,toc=listof, toc=bibnumbered, parskip=off, 11pt, a4paper]{tudreport}
	
	%TUD-Design Optionen
	% accentcolor, bigchapter & colorbacktitle: siehe Dokumentation zum TUD-Design
	% ACHTUNG: die accentcolor Einstellung entfaltet derzeit keine Wirkung, da am Ende dieser Datei eine modifizierte Identbar für die Kopfzeile eingebunden wird (um die zweifarbige Identbar in der Kopfzeile zu realisieren). Für eine einfarbige Identbar kann der Input-Befehl auskommentiert werden, wodurch accentcolor wieder seine Wirkung zurück erlangt.
	
	%KOMA-Script Optionen (werden durchgereicht an scrreprt)
	% 11pt								% setzt die Schriftgröße auf 11pt
	% a4paper							% Seitengröße DIN A4
	% toc=listof 							% sorgt dafür, dass die Verzeichnisse der Gleitumgebungen, z.B. das Abbildungs- und das Tabellenverzeichnis, im Inhaltsverzeichnis erscheinen (alternativ mit Nummerierung im Inhaltsverzeichnis: listofnumbered)
	% toc=bibnumbered  					% sorgt dafür, dass das Literaturverzeichnis einen nummerierten Eintrag im Inhaltsverzeichnis erhält (alternative Einstellung: ohne Nummerierung im Inhaltsverzeichnis mit Befehl bib)
	% parskip = full 						% setzt den Zeilenabstand auf eine volle Zeile und unterbindet den Erstzeileneinzug eines neuen Absatzes
	
	
	% % % % % % % % % % % % % % % % % % % % % % % % % % % % % % % % %
	%																		    %
	%	 Laden von Paketen zur Erweiterung des Funktionsumfangs						    %
	% 	Die Dokumentationen der verwendeten Pakete befinden im Unterordner ASFDKFAKDF	%
	%																		    %
	% % % % % % % % % % % % % % % % % % % % % % % % % % % % % % % % % 
	%\usepackage[ngerman]{babel}		
	% Das Babel-Paket passt LaTeX für die Verwendung der deutschen Sprache (neue deutsche Rechtschreibung) an.
	\usepackage[utf8]{inputenc} 
	% Eingabekodierung auf utf8 setzen (ermöglicht die direkte Verwendung von ä, ö, ü und ß beim Schreiben des Textes). Die Speicherung der TeX-Dateien muss im LaTeX-Editor entsprechend dieser Codierung erfolgen!
	\usepackage[T1]{fontenc}
	% Font-Encoding auf T1 setzen (stellt sicher dass ä, ö, ü und ß später im PDF gefunden werden)
	\usepackage{geometry}
	% Das Geometry-Paket erleichtert die Festlegung des Seitenlayouts und ermöglicht die einfache Verwendung mehrerer Layout-Umgebungen
	\usepackage{setspace}
	% Ermöglicht einfache Kontrolle über den Zeilenabstand mittels spacing-Befehlen und spacing-Umgebungen
	\usepackage{graphicx}		
	% ermöglicht die Einbindung von Grafiken und Bildern
	\usepackage{hyperref}											
	% erzeugt in der PDF-Ausgabe klickbare Links für Überschriften, etc. und setzt Lesezeichen welche beispielsweise in Adobe Reader genutzt werden können
	\usepackage{cleveref}
	\usepackage{booktabs}											
	% ermöglicht die Gestaltung professioneller Tabellen
	\usepackage{longtable}
	% ermöglicht die Gestaltung von Tabellen, welche mehrere Seiten umfassen, z.B. Symbolverzeichnisse (kompatibel zu booktabs)
	\usepackage[section]{placeins}									
	% verhindert dass Gleitumgebungen über Section-Grenzen hinaus wandern, führt außerdem den Befehl \FloatBarrier ein
	\usepackage{siunitx}
	% Paket zur korrekten und einheitlichen Darstellung von Zahlen und Einheiten (Verwendung siehe Dokumenation)
	\usepackage[babel, german=quotes]{csquotes} 
	\RequirePackage[backend=biber, style=ieee, citestyle=ieee, bibstyle=ieee, firstinits=true, maxnames=3%authortitle-icomp
	]{biblatex} %backend=biber
	\bibliography{main}
	% Einstellung der Anführungszeichen bei Zitaten
	%\usepackage[backend=biber, style=numeric-comp, bibstyle=numeric, citestyle=numeric]{biblatex}
	%\addbibresource{bibliography.bib}
	\usepackage{graphicx}
	%\usepackage{subfigure}
	\usepackage[format=plain, indention=0.5cm, justification=justified]{caption}
	\usepackage{subfigure}
	\usepackage{yhmath}
	\usepackage{amsmath}
	\usepackage{listings} 
	\usepackage[justification = centering]{caption}
	%	\usepackage{amssymb}
	\usepackage{color}
	\newcommand{\todo}[1]{\textcolor{red}{@TODO: #1}}
	\newcommand{\plusminus}[2]{$\pm$}
	\usepackage[mathcal]{eucal}
	\usepackage{pgfplots}
	\pgfplotsset{compat=1.12} %width=7cm,
	\usepackage{pgfplotstable}
%	\pgfplotsset{width=7cm,compat=1.13}
	%	%\usepgfplotslibrary{external} 
	%	%\usepgfplotslibrary[external]
	%	%\usetikzlibrary{pgfplots.external}
	%	%\usetikzlibrary[pgfplots.external]
	%	%\usepgfplotslibrary{external}
	%	%\tikzexternalize
	%	%\tikzset{external/force remake}
	%	
	\usepackage{multirow}
	\usepackage{rotating}
	\newcommand{\argmax}{\operatornamewithlimits{argmax}}
	\usepackage{listings}
	\usepackage{xcolor}
	
	\lstdefinestyle{sharpc}{language=[Sharp]C}
	
	
	% biblatex ist ein Paket zur Ausgabe und Formatierung von Zitaten und von Literaturverzeichnissen
	%backend=biber		% Einstellung für die Verwendung von biber als Bibliographie-Prozessor
	%style=numeric-comp		% Einstellung des
	%bibstyle=numeric			% Einstellung des Bibliographie-Stils
	%citestyle=numeric		% Einstellung des Zitierstils
	\usepackage[
	nomain,
	%xindy,
	nonumberlist, 			% keine Anzeige von Seitenzahlen der Einträge im Abkürzungs-/Symbolverzeichnis
	nogroupskip, 			% keine Gruppierung der Einträge nach Anfangsbuchstaben & keine zusätzlicher vertikaler Abstand bei Änderung des Anfangsbuchstabens
	acronym, 			% ein Abkürzungsverzeichnis erstellen
	toc			% Einträge im Inhaltsverzeichnis für das Abkürzungs- und Symbolverzeichnis
		% Übersetzung von Überschriften, etc. ins Deutsche
	]{glossaries}			% Paket zur Erstellung von Abkürzungs-, Symbolverzeichnissen, etc.
	\usepackage{blindtext}		% Ermögicht die einfache Verwendung von Beispieltext zu Testzwecken
	\usepackage[xindy]{imakeidx}
	\makeindex
	% weitere möglicherweise interessante Pakete: capt-of, multirow
	
	
	% % % % % % % % % % % % % % % % % % % 
	%										 %
	%	Festlegen von Einstellungen der Pakete		%
	%										 %
	% % % % % % % % % % % % % % % % % % % 
	
	\geometry{top=2.5cm, left=2.5cm, right=2.5cm, bottom=2cm}			
	% Definition der Seitenränder
	\graphicspath{{img/}}															% legt Unterverzeichnisse fest in denen \includegraphics standardmäßig nach Bilddateien sucht
	\sisetup{
		exponent-product = \cdot,		% setzt einen Malpunkt als Trennzeichen bei der Verwendung von Exponenten
		output-decimal-marker = {,},		% setzt Komma als Dezimalzeichen (im Englischen ist es ein Punkt)
		per-mode = symbol,			% legt fest, dass Einheiten im Nenner mit einem Bruchstrich statt mit negativen Exponenten dargestellt werden
		bracket-unit-denominator = false,	% mehrere Einheiten im Nenner werden nicht durch eine Klammer umschlossen
		range-phrase = ~--~,			% bei Verwendung von \SIrange wird ein Bindestrich zwischen den beiden Zahlen verwendet
		number-unit-product = \text{~},	% legt den Abstand zwischen Zahl und Einheit fest
		detect-all,					% hiermit werden Veränderungen der Schrift auch bei der Darstellung der Zahlen und Einheiten berücksichtigt
		list-final-separator = ~und~,		% Übersetzung
		list-pair-separator = ~und~,		% Übersetzung
	}
	\urlstyle{same}
	\hypersetup{ %
		colorlinks = true,		% true: stellt Verknüpfungen wie Links, Zitate und Verweise farbig dar, false: rahmt Links, Zitate und Verweise in einen farbigen Rahmen ein
		urlcolor = darkgray,		% Schriftfarbe für URLs
		citecolor = darkgray,		% Schriftfarbe für Zitate
		pdfborder = 0 0 1,		% legt Rahmen fest, sofern verwendet
		linkbordercolor = white,	% Rahmenfarbe bei Verweisen (white = transparent bei weißem Hintergrund)
		urlbordercolor = black,		% Rahmenfarbe bei URLs
		citebordercolor = white,		% Rahmenfarbe bei Zitaten (white = transparent bei weißem Hintergrund)
		bookmarksopen = false,		% legt fest ob die Lesezeichenleiste beim Öffnen des PDFs expandiert ist oder nicht
		bookmarksnumbered = true,	% legt fest ob die Kapitelnummern in den Lesezeichenbaum übernommen werden
	}
	
	% % % % % % % % % % % % % % % % % % %
	%										 %
	% 	Anpassung des Literaturverzeichnisses		%
	%										 %
	% % % % % % % % % % % % % % % % % % % 
	
	% läd die Literaturdatenbank aus dem aktuellen Arbeitsverzeichnis
	\ExecuteBibliographyOptions{%
		url = false,
		%dashed=false			%in Bibliographie-Stilen, welche das Literaturverzeichnis nach Autorennamen sortieren werden bei setzen der Einstellung auf true bei mehreren Quellen eines Autors die Einträge untereinander gruppiert und der Autorenname durch einen Strich ersetzt
		bibencoding=utf8, % wenn .bib in utf8, sonst ascii
		bibwarn=true, % Warnung bei fehlerhafter bib-Datei
		sorting=none % gibt Einträge im Literaturverzeichnis in der Reihenfolge aus, in der sie zitiert wurden	
	}%
	
	\DefineBibliographyStrings{ngerman}{%
		bibliography={Literaturverzeichnis},			% setzt die Überschrift des Literaturverzeichnis
		urlseen          = {Zugriff\addcolon}				% ändert die Beschriftung des Datums bei URLs von "besucht am" auf "Zugriff:"
	}
	
	\DeclareNameAlias{default}{last-first}  			
	%Im Literaturverzeichnis folgt die Darstellung der Autorennamen der Darstellung Nachname, Vorname (abhängig vom Stil, muss dies angepasst werden um die Effekt zu erhalten, siehe: http://projekte.dante.de/DanteFAQ/BiblatexReihenfolgeAutoren)
	\renewcommand*{\mkbibnamelast}[1]{\textsc{#1}}		
	%Setzen des Nachnamens des Autors in Kapitälchenschrift
	
	
	% % % % % % % % % %% % % % % % % % % % % % % % % 
	%												              %
	%	Neu-Definitionen und Änderungen bestehender Befehle		%
	%													      %
	% % % % % % % % % % % % % % % % % % % % % % % % % 
	
	% Neudefinition der Itemize-Umgebung um die Abstände zwischen einzelnen Stichpunkten von Aufzählungen zu Verringern
	\let\olditemize\itemize
	\renewcommand{\itemize}{
		\olditemize
		\itemsep4pt
	}
	
	% Neudefinition der Enumerate-Umgebung um die Abstände zwischen einzelnen Stichpunkten von Aufzählungen zu Verringern
	\let\oldenumerate\enumerate
	\renewcommand{\enumerate}{
		\oldenumerate
		\itemsep4pt
	}
	
	%\addto\extrasngerman{\def\figureautorefname{Abb.}}							
	% legt fest, dass bei Verweis auf eine Abbildung im Text, das Wort Abbildung mit Abb. abgekürzt wird
	%\addto\extrasngerman{\def\tableautorefname{Tab.}}							
	% legt fest, dass bei Verweis auf eine Tabelle im Text, das Wort Tabelle mit Tab. abgekürzt wird
	
	
	% Definition der Umgebung bibeinzug für manuelle Erstellung eines Eintrags des Literaturverzeichnisses
	\newenvironment{bibeinzug}{\hangindent=1cm}{\par}
	
	% Definition des Befehls Bibzahl für manuelle Erstellung eines nummerierten Eintrags im Literaturverzeichnis
	\newcommand{\bibzahl}[2] { %
		\begin{tabular}{p{0.75cm}p{\textwidth-1.5cm}}
			[#1] &#2
		\end{tabular}
	}	
