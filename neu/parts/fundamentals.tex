\chapter{Fundamentals}
\label{fundamentals}
	\section{Essential Equations}
		In the following, we will introduce the essential equations that form the basis of all methods and results discussed in this thesis. 
		\subsection{Equations of State}
		\label{EOS}
		First, we will enumerate the basic thermodynamic equations of state that describe the relations between the specific inner energy $e$, the local temperature $T$, the specific enthalpy $\bar{h}$, the pressure $p = p (\rho , e)$, the density $\rho$ and the specific entropy $s$.
		Assuming that all material parameters are constant, we obtain
		\begin{align}
			e &= c_v T \\
			\bar{h} &= c_p T = e + \dfrac{p}{\rho}
		\end{align}
		with the material parameters $c_p$ and $c_v$ as specific heat capacities at constant pressure and volume, respectively. Using these relations, we can define the heat capacity ratio 
		\begin{align}
			\gamma = \dfrac{c_p}{c_v} = \dfrac{\bar{h}}{e}, 
		\end{align}
		e.g. $\gamma = 1.4$ for standard air.
		
		Another essential equation is the relation for the specific entropy 
		\begin{align}
			T \, ds = de + p \, d\rho^{-1}.
		\end{align}
		
		\subsection{Ideal Gas Law}
		The equations of state mentioned above are not yet complete as there is missing a law for the pressure $p = p (\rho , e)$. In this thesis, we will limit ourselves to the ideal gas law as we are only modelling standard air ($\gamma = 1.4$). \\
		The ideal gas law is defined as
		\begin{align}
			p = (\gamma - 1) \rho e
			\label{idealgas}
		\end{align}
		with $\rho e \in \mathbb{R}^+$ denoting the inner energy. \\
		
		A definition of the speed of sound is given by
		\begin{align}
			a = \sqrt{\left.\dfrac{\partial p}{\partial \rho}\right|_s}.
		\end{align}
		Using this definition combined with the ideal gas law \eqref{idealgas} and an equation for the change of entropy for isentropic flows, as found in \textcite{mueller2014}, leads to
		\begin{align}
			a^2 &= \gamma \dfrac{p}{\rho}.
		\end{align}
		For fully isentropic flows, on which we will be concentrating, it also follows that 
		\begin{align}
				\dfrac{p}{\rho^\gamma} &= \text{const}.
		\end{align}
		
		\subsection{Navier-Stokes Equations}
		The \gls{cns} equations in conservative forms read as 
		\begin{align}
			\dfrac{\partial \mathbf{U}}{\partial t} + \dfrac{\partial \mathbf{F}_i^c(\mathbf{U})}{\partial x_i} - \dfrac{\partial \mathbf{F}_i^v(\mathbf{U}, \nabla\mathbf{U})}{\partial x_i} = \mathbf{B},
		\end{align}
		with $\mathbf{U}$ as the conserved flow variables, $\mathbf{F}_i^c$ and $\mathbf{F}_i^v$ as the convective and viscous fluxes and $\mathbf{B}$ as source term:
		\begin{align}
			\mathbf{U} = 
				\begin{pmatrix}
				\rho \\
				\rho v_j \\
				\rho E
				\end{pmatrix} , \quad
			\mathbf{F}_i^c = 
				\begin{pmatrix}
				\rho v_i \\
				\rho v_i v_j + p \delta_{ij}\\
				v_i(\rho E + p)
				\end{pmatrix} , \quad
			\mathbf{F}_i^v = 
				\begin{pmatrix}
				0 \\
				\tau_{ij}\\
				\tau_{ij} + q_i
				\end{pmatrix} , \quad
			\mathbf{B} = 
				\begin{pmatrix}
				0 \\
				\rho F_j\\
				\rho F_j v_j + Q_i
				\end{pmatrix}.
		\end{align}
		In addition to the denotations in \cref{EOS}, we have $F_j$ as body forces, $Q_i$ as heat sources, the viscous stress tensor
		\begin{align}
			\tau_{ij} = \mu \left[\left(\dfrac{\partial v_i}{\partial x_j} + \dfrac{\partial v_j}{\partial x_i} \right) - \dfrac{2}{3} \dfrac{\partial v_k}{\partial x_k} \delta_{ij}\right]
		\end{align}
		with the dynamic viscosity $\mu$ and the heat flux $q_i$ modelled using Fourier's Law
		\begin{align}
			q_i = k \dfrac{\partial T}{\partial x_i}.
		\end{align}
		
			\paragraph{Euler Equations}
			Regarding only compressible inviscid flow, the viscous fluxes and the source terms dissolve ($\mathbf{F}_i^v = \mathbf{B} = \mathbf{0}$) and the Navier-Stokes equations simplify to the so-called Euler equations: 
			\begin{align}
				\dfrac{\partial \mathbf{U}}{\partial t} + \dfrac{\partial \mathbf{F_x^c(U)}}{\partial x} + \dfrac{\partial \mathbf{F_y^c(U)}}{\partial y} = 0.
			\end{align}
	
	
	\section{Dimensionless Measures}
	For it is much easier to handle dimensionless \gls{pde}s, we will introduce some dimensionless measures as found in \textcite{annualreport}. In order to derive these, we need some reference quantities: $L_\infty$ as a reference length, a reference velocity $V_\infty$, reference density $\rho_\infty$, reference volume force $g_\infty$, reference viscosity $\mu_\infty$, reference thermal conductivity coefficient $k_\infty$ and the gas constant $R$. All other reference quantities can be derived from those. In the following, all dimensionless quantities will be marked with an asterisk $(\cdot)^*$
	
	\begin{align}
		t^* &= \dfrac{V_\infty}{L_\infty} \cdot t,& x_i^* &= \dfrac{1}{L} \cdot x_i, & v_i^* &= \dfrac{1}{V_\infty}\cdot v_i, & \rho^* &= \dfrac{1}{\rho_\infty} \cdot \rho, & p^* &= \dfrac{1}{\rho_\infty V_\infty^2} \cdot p, \\ \nonumber
		 \mu^* &= \dfrac{1}{\mu_\infty} \cdot \mu , & k^* &= \dfrac{1}{k_\infty} \cdot k,&  T^* &= \dfrac{R}{V_\infty^2}\cdot T, & F_j^*& = \dfrac{1}{g_\infty} \cdot F_j, & \rho E^* &= \dfrac{1}{\rho_\infty V_\infty^2} \cdot \rho E, \\ \nonumber
		 Q_i^* &= \dfrac{L}{V_\infty^3} \cdot Q. &
	\end{align}
	
	In order to derive the non-dimensional form of the \gls{cns} equation, we also need the dimensionless operators
	
	\begin{align}
		\frac{\partial}{\partial t}= \frac{\partial t^*}{\partial t} \frac{\partial}{\partial t^*} &= \frac{V_\infty}{L} \frac{\partial}{\partial t^*}, \\
		\frac{\partial}{\partial x_i} = \frac{\partial x_i^*}{\partial x_i} \frac{\partial}{\partial x_i^*} &= \frac{1}{L} \frac{\partial}{\partial x_i^*}, \\
		\nabla &= \frac{1}{L} \nabla^*.
	\end{align}
	
		
	\subsection{Non-dimensional Ideal Gas Law}
	For a closed system of equations, we have to use the ideal gas law in dimensionless form: 

	\begin{gather}	
		\begin{aligned}
			p^* &= \rho^* (\gamma - 1) e^* \\ 
			&= (\gamma - 1) \left(\rho E^* - \dfrac{1}{2} \rho^* \mathbf{v}^{*2}\right) 
		\end{aligned}
	\end{gather}
	\subsection{Dimensionless Navier-Stokes Equations}
	As we now have all required measures and operators, we can use them to define the dimensionless relations 
	\begin{description}
		\item[Reynolds Number] $\text{Re} = \dfrac{\rho_\infty V_\infty L}{\mu_\infty} \propto \dfrac{\text{inertia forces}}{\text{viscous forces}}$,
		\item[Froude Number] \quad $\text{Fr} = \dfrac{V_\infty }{\sqrt{g L}} \propto \dfrac{\text{body inertia}}{\text{gravitational forces}}$,
		\item[Prandtl Number]\quad $\text{Pr} = \dfrac{ \mu_\infty c_p}{k_\infty} \propto \dfrac{\text{viscous diffusion rate}}{\text{thermal diffusion rate}}$.
	\end{description}
	Putting all together, we receive the dimensionless Navier-Stokes equations. As they only depend on non-dimensional quantities, we can drop the asterisk:
	\begin{align}
		\dfrac{\partial \mathbf{U}}{\partial t} + \dfrac{\partial \mathbf{F}_i^c(\mathbf{U})}{\partial x_i} - \dfrac{\partial \mathbf{F}_i^v(\mathbf{U}, \nabla\mathbf{U})}{\partial x_i} = \mathbf{B}
	\end{align}
	
	with the dimensionless fluxes
	\begin{align}
		\mathbf{F}_i^c = 
		\begin{pmatrix}
			\rho v_i \\
			\rho v_i v_j + p \delta_{ij}\\
			v_i(\rho E + p)
		\end{pmatrix} , \quad
		\mathbf{F}_i^v = \dfrac{1}{\text{Re}}
		\begin{pmatrix}
			0 \\
			\tau_{ij}\\
			\tau_{ij}v_j + \dfrac{\gamma}{\text{Pr}(\gamma-1)} q_i
		\end{pmatrix} , \quad
		\mathbf{B} = \dfrac{1}{\text{Fr}^2}
		\begin{pmatrix}
			0 \\
			\rho F_j\\
			\rho F_j v_j
		\end{pmatrix}
		+ 
		\begin{pmatrix}
			0 \\
			0\\
			Q_j
		\end{pmatrix}
		.
	\end{align}	
	