%Ein extra Verzeichnis für Symbole erstellen
\newglossary[slg]{symbolslist}{syi}{syg}{Symbolverzeichnis}

%Den Punkt am Ende jeder Beschreibung deaktivieren
\renewcommand*{\glspostdescription}{}

%Glossar-Befehle anschalten
\makeglossaries

%				    	    %
%	Symbole definieren	%
%					    %
\newglossaryentry{symb:epsilon}{
	name=$\varepsilon$,
	description={Dehnung},
	sort=epsilon, type=symbolslist
}
\newglossaryentry{symb:phi}{
	name=$\varphi$,
	description={Winkel},
	sort=phi, type=symbolslist
}
\newglossaryentry{symb:E}{
	name=E,
	description={Elastizitätsmodul},
	sort=E, type=symbolslist
}
\newglossaryentry{symb:T}{
	name=T,
	description={Temperatur},
	sort=T, 
	type=symbolslist,
	symbol=\si{\kelvin}
}
\newglossaryentry{symb:sigma}{
	name=$\sigma$,
	description={Spannung},
	sort=sigma, 
	type=symbolslist,
	symbol=\si{\newton\per\millimetre\squared}
}
%							%
%	Abkürzungen definieren		%
%							%
%Befehls-Schema:
%\newacronym{label}{Abkürzung}{Bedeutung}
\newacronym{tud}{TUD}{Technische Universität Darmstadt}
\newacronym{spz}{SPZ}{Sprachenzentrum}
\newacronym{pmv}{PMV}{Fachgebiet Papierfabrikation und Mechanische Verfahrenstechnik}


