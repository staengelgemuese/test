\chapter*{Symbolverzeichnis}
\addcontentsline{toc}{chapter}{Symbolverzeichnis (tabellarische Alternative)}
\begin{spacing}{1.5}
\setlength\LTleft{15pt}
\setlength\LTright{\fill}
\begin{longtable}{@{\extracolsep{\fill}}lll}
    \textbf{Symbol}	&\textbf{Einheit}		&\textbf{Beschreibung}                                           \\\midrule\endhead
    \(D\)			&\si{\mm}				&Durchmesser                                                        \\
    \(D_0\)			&\si{\mm}                                           &Ausgangsdurchmesser                                                        \\
    \(\varepsilon\)                &                                           &Dehnung                                                        \\
    \(E\)                &\si{\N\per\mm\squared}                                           &Elastizitätsmodul                                                        \\
    \(G\)                &\si{\N\per\mm\squared}                                           &Schubmodul                                                        \\
    \(\lambda\)                &\si{\nm}                                           &Wellenlänge                                                        \\
    \(l\)                &\si{\mm}                                           &Länge                                                      \\
    \(l_0\)                &\si{\mm}                                           &Ausgangslänge                                                        \\
    \(\Delta l\)                &\si{\mm}                                           &Längendifferenz                                                        \\
    \(n\)                &                                           &Anzahl der Versuche                                                        \\
    \(\omega\)                &\si{\per\s}                                           &Kreisfrequenz                                                        \\
    \(\sigma\)                &\si{\N\per\mm\squared}                                           &Spannung                                                                                                           \\
    \(t\)                &\si{\s}                                           &Zeit                                                        \\
    \(t_0\)                &\si{\s}                                           &Ausgangszeitpunkt                                                        \\
    \(\Delta t\)                &\si{\s}                                           &Zeitdifferenz                                                        \\
    \(T\)                &\si{\degreeCelsius}                                           &Temperatur                                                        \\
    \(T_0\)                &\si{\degreeCelsius}                                           &Ausgangstemperatur bzw. Bezugstemperatur                                                        \\
    \(\Delta T\)                &\si{\degreeCelsius}                                         &Temperaturdifferenz                                                        \\
    \O                &                                           &Durchschnitt                                                        \\
\end{longtable}
\end{spacing}

                    