\HeaderOne{Boundary fluxes}
\label{ch:boundaryFluxes}
In this chapter we analyze and discuss the boundary of the simulation domain in more detail.
We explicitly set the flow at each boundary edge for, e.g., simulating a wind tunnel.
As our velocity form is divergence free, the sum of the boundary fluxes has to be zero.
Lets examine the three steps of our algorithm (see Section \ref{sec:SimpleAlgorithm}).
The first step, i.e., vorticity advection, can easily be adapted to boundary fluxes by handling particles which leave the simulation domain. 
This can be solved in many ways, e.g., by clamping the particles to the simulation domain.
In the second step, we add the external forces.
External forces are composed of boundary fluxes (harmonic part) and inner vorticities.
In the velocity recovery step, we implicitly add the harmonic external forces by defining appropriate boundary conditions for the associated equations.

\paragraph*{}
In the first part of this chapter, we derive a unique characterization of the velocities in terms of inner vorticities and given boundary fluxes.
Based on this characterization, we construct the equations which we need to solve to recover the velocities.

\HeaderTwo{Vorticity based characterization of the velocity}
\label{sec:VorticityRepresentation}
Now, we will prove that a divergence free velocity form $\VelocityForm$ is uniquely characterized through the inner vorticities and given boundary fluxes.
We assume, that the domain is contractible and denote it by $K$.
The vector space of all divergence free 1P-forms in $K$ is defined as:

\begin{equation}
R := \ker d_1 = \{ \VelocityForm \in \DiffForms^P_1(K) : d_1 \VelocityForm = 0 \}
\end{equation}

Let $N \in \N$ be the number of boundary edges in the complex $K$ and suppose $\Gamma_B : R \rightarrow \R^N$ is a projective mapping s.t. ${\Gamma_B}_j$ maps the flux of the j-th boundary edge to $\R$.
$\Gamma_B$ is a linear function and describes the boundary flux configuration for a given divergence free velocity form $\VelocityForm \in R$. 
%The curl operator has been defined in Section \ref{sec:FluidEquations} as $C := d_0^T *_1$.
Similarly to the boundary fluxes, we extract the inner vorticities with a linear function
$\Gamma_I : R \rightarrow \R^M$, where $M \in \N$ is the number of inner dual faces.
Combining both linear operators leads to the following definition:

\begin{equation}
\bigotimes : R \rightarrow \R^N \times \R^M , \bigotimes(\VelocityForm) := (\Gamma_B(U),\Gamma_I(U))
\end{equation}

$\bigotimes$ maps linearly a given velocity form to its \newword{configuration space} $\R^N \times \R^M$.
Our uniqueness statement is true if and only if $\bigotimes$ is injective, which is equivalent to $\ker \bigotimes = \{0\}$.

\newpage
\HeaderThree{Proof}
We prove the theorem by contradiction. Suppose there exists a velocity form $\VelocityForm \in \ker \bigotimes$ s.t. $\VelocityForm \neq 0$. 
That means, $\VelocityForm$ has no boundary fluxes and no interior vorticities but contains inner fluxes.
We will show, that there exists a dual loop whose circulation is not zero. Since the topology is contractible,
every dual loop is the boundary of a chain of dual faces (see Figure \ref{fig:DualLoop}).
By Stokes Theorem, there exists a dual face inside the dual loop whose vorticity is non-zero. 
But this contradicts the assumption that all the inner vorticities in $\VelocityForm$ are zero, i.e., $\Gamma_I(\VelocityForm)=0$.
It remains to construct the desired dual loop.
A face with an incoming flow must have an outgoing and vice versa, as the velocity form is divergence free.
%A face with an incoming flow contains an outgoing flow and likewise exists an incoming flow if there is an outgoing flow. 
This is a consequence of the divergence free property and the basis for our dual loop construction.
Let $f_0$ be a primal face which contains an outgoing and therefore an incoming flow. 
Choose a boundary edge $e_0$ of $f_0$ with outgoing flow and assume $d_0$ to be its associated dual edge. 
We leave the face $f_0$ through the edge $e_0$ and enter a new face.
We denote that face with $f_1$.
Again, $f_1$ contains an incoming and thus an outgoing flow.
We cannot leave the domain as we have zero boundary fluxes.
Therefore, we eventually encounter an already visited face if we repeat the process.
The resulting sequence of dual edges $d_0 , ... , d_n$ spans a dual loop.
Since we moved along outgoing flows, the total circulation of the dual loop is positive.
\begin{flushright}
$\square$
\end{flushright}

\begin{figure}[h!]
	\begin{minipage}[b]{0.5 \linewidth}
		\centering
		\subfigure[]{
			      \input{pics/tikz/regGridDualLoop}}
	\end{minipage}
	\begin{minipage}[b]{0.5 \linewidth}
		\centering
		\subfigure[]{
			      \input{pics/tikz/meshDualLoop}}
	\end{minipage}	
	\caption[A dual loop in a regular grid and simplicial complex]
			{(a) is a dual loop in a regular grid and (b) in a simplicial complex. The loops are the boundary of a chain of dual faces.}
	\label{fig:DualLoop}
\end{figure}

\newpage
\HeaderTwo{Implementing boundary conditions}
From the previous section, we know that a velocity form is uniquely described through inner vorticities and given boundary fluxes.
In this section, we will discuss how to construct a sparse symmetric system of linear equations which is based on the vorticity
and boundary flux configuration.

\HeaderThree{Some formal notations}
The different simplices in a complex are combined in the set:

\begin{equation}
\SimplicesTypes := \{ \SimplexFace,\SimplexEdge,\SimplexVertex \}
\end{equation}

where ($\SimplexFace$=Faces,$\SimplexEdge$=Edges,$\SimplexVertex$=Vertices).
We define three types of diagonal matrices for each simplex type, to distinguish boundary elements from interior ones.
For $\SimplexType \in \SimplicesTypes$ we define the \newdef{interior matrix} as:

\begin{equation}
\InteriorMatrix^\SimplexType_{ii} := \begin{cases} 1 & \text{if i-th simplex of type } \SimplexType \text{ is not a boundary element} \\
						  0 & \text{otherwise} \end{cases}
\end{equation}

Similarly, the \newdef{boundary matrix} is defined as:

\begin{equation}
\BoundaryMatrix^\SimplexType_{ii} := \begin{cases} 1 & \text{if i-th simplex of type } \SimplexType \text{ is a boundary element} \\
						  0 & \text{otherwise} \end{cases}
\end{equation}

For each simplex type $\SimplexType \in \SimplicesTypes$, we define an identity matrix and denote it by $\IdentityMatrix^\SimplexType$.
The equation $\IdentityMatrix^\SimplexType = \InteriorMatrix^\SimplexType + \BoundaryMatrix^\SimplexType$ is an obvious consequence of our defined matrices.

\HeaderThree{Constraints}
A divergence free velocity form $\VelocityForm \in \DiffForms^P_1(K)$, whose boundary fluxes are
$B_F \in \DiffForms^P_1(K)$ and its inner vorticities are equal to $\VorticityForm \in \DiffForms^D_2(K)$,
satisfies the following constraints:

\begin{itemize}
\item \textbf{Boundary fluxes: } $\BoundaryMatrix^\SimplexEdge \VelocityForm = B_F$
\item \textbf{Inner vorticities: } $\InteriorMatrix^\SimplexVertex \DECCurl \VelocityForm = \InteriorMatrix^\SimplexVertex \VorticityForm$
\item \textbf{Divergence free: } $d \VelocityForm = 0$
\end{itemize}


\HeaderThree{Sparse symmetric system of linear equations}
In Section \ref{sec:VortToVel}, we combined the \textbf{inner vorticities} and \textbf{divergence free} constraints by searching a \newword{potential} $\Phi$
of $\VelocityForm$. We modify the constraints in order to implement this approach and end up with the following equations:

\begin{eqnarray}
\label{eq:BoundaryFluxes} 
\BoundaryMatrix^\SimplexEdge d \Phi = B_F \\
\label{eq:InnerVorticityPotential}
\InteriorMatrix^\SimplexVertex \DECLaplace \Phi =\InteriorMatrix^\SimplexVertex \VorticityForm \\
\VelocityForm = d \Phi
\end{eqnarray}

$\BoundaryMatrix^\SimplexEdge d \Phi = B_F$ are Neumann-boundary conditions w.r.t. the exterior derivative operator $d$.
The operator $\DECLaplace$ has been defined in Section \ref{sec:VortToVel} as $\DECLaplace := \DECCurl d$.
Instead of directly computing the velocity form $\VelocityForm$, we need to compute a potential $\Phi$.
We change the Neumann-boundary conditions into Dirichlet boundary conditions by eliminating the derivative operator in \myeqref{eq:BoundaryFluxes}.
For a 0P-form $\Theta \in \DiffForms_0^P(K)$ which satisfies

\begin{equation}
\BoundaryMatrix^\SimplexEdge d \Theta = B_F
\end{equation}

we can reformulate \myeqref{eq:BoundaryFluxes} as

\begin{equation}
\label{eq:BForcePotential}
\BoundaryMatrix^\SimplexVertex \Phi = \BoundaryMatrix^\SimplexVertex \Theta
\end{equation}

$\Theta$ can be computed by integrating $B_F$ on the boundary.
\myeqref{eq:BForcePotential} establishes our Dirichlet boundary conditions.
\myeqref{eq:BForcePotential} and \myeqref{eq:InnerVorticityPotential} can be combined as
the images of $\InteriorMatrix^\SimplexVertex \DECLaplace$ and $\BoundaryMatrix^\SimplexVertex$ are distinct. 
Hence, we get:

\begin{eqnarray}
(\InteriorMatrix^\SimplexVertex \DECLaplace + \BoundaryMatrix^\SimplexVertex) \Phi = \InteriorMatrix^\SimplexVertex \VorticityForm + \BoundaryMatrix^\SimplexVertex \Theta \\
\VelocityForm = d \Phi
\end{eqnarray}

But, $\InteriorMatrix^\SimplexVertex \DECLaplace + \BoundaryMatrix^\SimplexVertex$ is not symmetric because of $\InteriorMatrix^\SimplexVertex \DECLaplace$. 
Therefore, it is not suited for an efficient solver like CG.
We use the identity $\IdentityMatrix^\SimplexVertex = \InteriorMatrix^\SimplexVertex + \BoundaryMatrix^\SimplexVertex$ to split
$\InteriorMatrix^\SimplexVertex \DECLaplace$ into the term $\InteriorMatrix^\SimplexVertex \DECLaplace \InteriorMatrix^\SimplexVertex + \InteriorMatrix^\SimplexVertex \DECLaplace \BoundaryMatrix^\SimplexVertex$
and moreover we know
$\InteriorMatrix^\SimplexVertex \DECLaplace \BoundaryMatrix^\SimplexVertex \Phi = \InteriorMatrix^\SimplexVertex \DECLaplace \BoundaryMatrix^\SimplexVertex \Theta$.
Finally, \myeqref{eq:VortVelFinished} is a symmetric sparse system of linear equations.

\begin{equation}
\label{eq:VortVelFinished}
(\InteriorMatrix^\SimplexVertex \DECLaplace \InteriorMatrix^\SimplexVertex + \BoundaryMatrix^\SimplexVertex) \Phi = \InteriorMatrix^\SimplexVertex \VorticityForm + \BoundaryMatrix^\mathcal{V} \Theta - \InteriorMatrix^\SimplexVertex \DECLaplace \BoundaryMatrix^\SimplexVertex \Theta
\end{equation}

We solve the equation for $\Phi$ and get the velocity form by computing $d \Phi$.

\newpage

Now, we can explicitly set the flow which goes through the boundary of our simulation domain.
The simulation domain is not closed anymore and we are able to simulate for example a wind tunnel.