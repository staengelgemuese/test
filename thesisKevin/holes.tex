\HeaderOne{Arbitrary topologies}
\label{ch:Topology}
In this chapter, we extend the basic algorithm to domains with a non-contractible topology, i.e., holes inside the domain.
The steps $"$vorticity advection$"$ and $"$adding external forces$"$ can easily be adapted to arbitrary topologies.
We will see in Section \ref{sec:VorticityRep} that inner vorticities and boundary fluxes are not
sufficient to uniquely determine a divergence free velocity field.
Therefore, we analyze the difference between contractible and non-contractible topologies in more detail.
The representation has to be modified in order to cover non-contractible topologies.
In the second part of this chapter, we extend our algorithm to arbitrary domains, based on this new characterization.

\HeaderTwo{Vorticity representation with holes}
\label{sec:VorticityRep}

Figure \ref{fig:InfiniteManySolutions} outlines two examples of a divergence free velocity form with no boundary flow and no inner vorticities.
As both velocity forms are obviously different, inner vorticities and boundary fluxes are not sufficient to uniquely describe a velocity form.

\begin{figure}[h!]
\label{fig:InfiniteManySolutions}
	\begin{minipage}[b]{0.5 \linewidth}
		\centering
		\subfigure[A flow of 1]{
			      \input{pics/tikz/regGridHoleLowFlow}}
	\end{minipage}
	\begin{minipage}[b]{0.5 \linewidth}
		\centering
		\subfigure[A flow of 2]{
			      \input{pics/tikz/regGridHoleHighFlow}}
	\end{minipage}	
\caption[two different velocity forms without boundary fluxes and inner vorticities]
		{We defined two different velocity forms without boundary fluxes and inner vorticities.
			  The domain does not have a contractible topology as it has a hole in its center.}
\end{figure}

Our characterization for contractible domains is based on the fact that inner vorticities uniquely describe the circulations along all dual loops.
Therefore, it is sufficient to know the vorticities in order to derive the circulations.
This connection is established through Stokes theorem, as every loop is the boundary of a chain.
But, a loop around a hole cannot be the boundary of a chain and thus it is not applicable.
Therefore, we have to treat circulations around holes explicitly.
However, the theorem enables transforming two different loops around the same hole into each other.
Hence, it is sufficient to know the circulation of one loop for each hole.
Our characterization can then be stated as:

\begin{center}
A velocity field is uniquely described through its inner vorticities, boundary fluxes and circulations around holes.
\end{center}

\myparagraph{Proof:}
In the previous chapter, we introduced a linear function which maps a velocity form to its configuration space.
The formal shape of that function was $\bigotimes : R \rightarrow \R^N \times \R^M$, where $R$ identifies $\ker d_1$.
We proved the uniqueness by showing, that the kernel of $\bigotimes$ is trivial.
Now, $\bigotimes$ has to be modified to take circulations around holes into account.
Let us denote the number of holes by $N_L \in \N$. We can linearly map the circulations around holes by a function
$\Gamma_H : R \rightarrow \R^{N_L}$, where ${\Gamma_H}_j$ is the circulation around the j-th hole.
We combine the functions $\bigotimes$ and $\Gamma_H$ to a function $\bigotimes'$:

\begin{equation}
\bigotimes' : R \rightarrow \R^N \times \R^M \times \R^{N_L}, \bigotimes'( \VelocityForm ) := (\bigotimes(\VelocityForm),\Gamma_H(\VelocityForm))
\end{equation}

Like in the previous chapter, the new characterization is unique if and only if $\ker \bigotimes' = \{ 0 \}$.
If we have a non-zero velocity form $\VelocityForm \in R$ with zero boundary fluxes, there exists a dual loop with non-zero circulation. 
Again, we can create a dual loop with positive circulation, as the construction has not been dependent on the actual topology.
Since the dual loop has a positive circulation, there has to be a dual face, or dual loop around a hole, with non-zero vorticity or circulation, respectively. 
\begin{flushright} $\square$ \end{flushright}

\HeaderTwo{Holes in a domain}
\label{sec:HolesDomain}
We consider a simplicial complex or regular grid denoted by $K$.
The \newword{symmetric positive definite(spd)} equations to recover the velocity (see previous chapter) are:

\begin{equation}
\label{eq:VortVelFinished2}
(\InteriorMatrix^\SimplexVertex \DECLaplace \InteriorMatrix^\SimplexVertex + \BoundaryMatrix^\SimplexVertex) \Phi = \InteriorMatrix^\SimplexVertex \VorticityForm + \BoundaryMatrix^\mathcal{V} \Theta - \InteriorMatrix^\SimplexVertex \DECLaplace \BoundaryMatrix^\SimplexVertex \Theta
\end{equation}

The velocity is recovered by computing the derivative of $\Phi$.
We have to modify \myeqref{eq:VortVelFinished2} in order to adapt the generalized characterization.
However, instead of modifying the equations, we modify the domain. 
We create a second domain by contracting each hole to a single vertex as illustrated in Figure \ref{fig:ContractingTriangle}.

\begin{figure}[h!]
\centering
\includegraphics[width=9cm,height=3cm]{pics/pdf/ContractingTriangle.pdf}
\caption[Contracting a triangle]
		{All boundary edges of a single hole are contracted to a vertex.
		      By contracting the holes, we create a new domain with a contractible topology.}
\label{fig:ContractingTriangle}
\end{figure}

We call the new domain \newword{contracted domain} and denote it by $K'$.
As the name suggests, the contracted mesh has a contractible topology. 
Therefore, we can define and solve our equations on the new domain.
Every hole in $K$ is mapped to a single vertex in $K'$, respectively. 
The idea is to represent the circulation around a hole as vorticity defined on its associated vertex in $K'$.
In the absence of external forces, the circulation around a hole does not change.
That means that the circulation is specified through the initial state of the simulation and external forces.

\paragraph*{}
In a first step the vorticities are mapped from $K$ to $K'$.
Moreover, we map circulations around holes in $K$ to their respective vorticities in $K'$.
As the topology on $K'$ is contractible, we can recover the velocity on $K'$.
Finally, we map back the velocity to $K$.
\begin{center}
$K \xrightarrow{\text{Transfer vorticity and circulations}} K' \xrightarrow{\text{Recover Velocity}}K' \xrightarrow{\text{Transfer Velocity}} K$
\end{center}

\myparagraph{Formal definition of the contraction}
First, we will give a formal definition of the contraction process.
The vertices of $K$ and $K'$ are identified by $V$ and $V'$, respectively.
$V'$ is composed of the inner and outer boundary vertices of $K$.
Furthermore, we add a single vertex for each hole in $K$ to $V'$.
The vertices of $K$ and $K'$ are connected with a function $F_V : V \rightarrow V'$.
Two vertices $v_0 \neq v_1 \in V$ are said to be \newword{contracted} if
$F_V(v_0) = F_V(v_1)$.
The \newword{contraction} relation is obviously an equivalence relation.
Each non-trivial equivalence class corresponds to exactly one hole in $K$.
We formally define the set of the corresponding vertices in $K'$ as:

\begin{equation}
L(K') := \{ v \in V' : | {F_V}^{-1}(v) | > 1 \}
\end{equation}

The edges of both domains are denoted as $E$ and $E'$. 
Every edge in $E$ is spanned by two vertices $v_0, v_1 \in V$.
An edge $e \in E$ is called \newword{contracted} if $F_V(v_0) = F_V(v_1)$.
$E'$ is induced by the remaining non-contracted edges of $E$.
We connect the edges of $E$ with those of $E'$ by introducing a function
$F_E : E \rightarrow E' \cup \{0\}$. 
Likewise, only the boundary edges of inner holes are contracted.
Hence, the flux through a contracted edge is $0$.
For all non-contracted edges $e \in E$, $F_E$ maps to the corresponding edge in $K'$.
For all contracted edges, $F_E$ maps to $0$.

\myparagraph{Vorticity transfer operator}
Let us denote our initial vorticity as $\VorticityForm \in \DiffForms^D_2(K)$.
We are only interested in inner vorticities and thus eliminate the boundary vorticities:

\begin{equation}
\VorticityForm \leftarrow \InteriorMatrix^\SimplexVertex \VorticityForm
\end{equation}

For each inner hole, we set the vorticities of its boundary vertices to the associated circulation.
This modification implies the following relation:

\begin{equation}
F_V(v_0) = F_V(v_1) \Rightarrow \VorticityForm(v_0) = \VorticityForm(v_1)
\end{equation} 

The vorticity transfer operator is then defined as:

\begin{equation}
\Pi_\Omega : \DiffForms(K)^D_2 \rightarrow \DiffForms(K')^D_2, \Pi_\Omega( \Omega )(v') := \Omega( F_V^{-1}(v') )
\end{equation}

Intuitively, $\Pi_\Omega$ just copies the inner vorticities of $K$ onto the inner vertices of $K'$.
We defined the vorticities s.t. the assigned vorticity of $v \in L(K')$ is the circulation of its associated hole in $K$.

\myparagraph{Velocity transfer operator}
Two non-contracted edges $e \neq f \in E$ are said to \newword{merge} if $F_E(e) = F_E(f)$.
Edges are merged if and only if they are both boundary edges of the same face and all the remaining boundary edges of that face are contracted. 
Thus, for a divergence free velocity form, the oriented fluxes through merged edges are equal. 
$F_V$ induces an orientation onto the edges in $K'$.
This must be taken into account when mapping back the velocities.
We formally define the induced orientation through the following function:

\begin{equation}
O_E : E \rightarrow \{-1,0,1\}, O_E(e) :=
\begin{cases}
0 & \text{if } e \text{ is contracted} \\
1 & \text{if } F_V(v_0) \text{ is starting vertex of } F_E(e) \\
-1 & \text{if } F_V(v_0) \text{ is ending vertex of } F_E(e)
\end{cases}
\end{equation}

The velocity transfer operator is defined as:

\begin{equation}
\Pi_\VelocityForm : \DiffForms(K')^P_1 \rightarrow \DiffForms(K)^P_1, \Pi_\VelocityForm(\VelocityForm')(e) := 
\begin{cases}
0 & \text{if } e \text{ is contracted} \\
O_E(e) \cdot \VelocityForm'( F_E(e) ) & \text{else}
\end{cases}
\end{equation}

$\Pi_\VelocityForm$ is conservative w.r.t. divergence free forms.
I.e., if $\VelocityForm \in \DiffForms(K')^P_1$ is divergence free, then $\Pi_\VelocityForm(\VelocityForm)$ is also divergence free.

\myparagraph{Solving equation in K'}
If two vertices $v \in V$ and $v' \in V'$ are connected by $F_V$, i.e., $F_V(v) = v'$, their respective vorticities are equal.
Likewise, edges have the same oriented flux if they are connected by $F_E$.
In that sense, the velocities and vorticities defined in $K$ and $K'$ are equivalent.
Therefore, we need to ensure that the curl operators defined in $K$ and $K'$ are compatible.
That means, for curl operators $\DECCurl$ in $K$ and $\DECCurl'$ in $K'$ and forms $\VorticityForm \in \DiffForms^D_2(K)$ and $\VelocityForm' \in \DiffForms^P_1(K')$ we have:

\begin{equation}
\DECCurl'(\VelocityForm') = \Pi_\VorticityForm(\Omega) \Rightarrow \DECCurl(\Pi_\VelocityForm( \VelocityForm' )) = \Omega
\end{equation} 

Based on this compatibility constraint, we generate an uniquely defined hodge-star operator $*_1'$ on $K'$.
We introduced two versions of the hodge-star for edges in Subsection \ref{ssec:HodgeStar} and \ref{ssec:ModifiedHodge}.
The standard hodge-star is just a diagonal matrix which we adopted from \myciteS{elcott}.
Then, we created a modified version (smooth hodge-star) in Subsection \ref{ssec:ModifiedHodge} on regular grids to solve the compatibility issue between DEC operators and the velocity field reconstruction.
However, we first discuss the standard hodge star and afterwards, apply the results to the smooth hodge-star.

\HeaderThree{Simple hodge-star on edges}
Now, we define a new hodge-star in $K'$ which we will denote as $*_1'$.
The standard hodge-star operator $*_1$ is a diagonal matrix whose entries are terms of the following form:

\begin{equation}
*_{ii} := \frac{\text{Volume of dual edge $i$}}{\text{Volume of primal edge $i$}}
\end{equation}

If all incident edges of an inner primal vertex are neither contracted nor
merged, the curl operators in $K$ and $K'$ should be defined equivalently for that vertex.
However, as inner vertices may not have incident contracted edges, we only have to consider merged edges.
%\newpage

\myparagraph{Merged edges}

Figure \ref{fig:VorticityCollidingEdges} illustrates the incident edges of primal vertices $v$ and $v'$ before and after merging.% in $K$ and $K'$, respectively.
The edges $e_1$ and $e_6$ in that figure are merged to $e_{6,1}'$.
The flux through some edge $e_\mu$ is identified by $\VelocityForm(e_\mu)$ and its corresponding hodge-star entry as $*_1(e_\mu)$.
The vorticity at vertex $v$ in Figure \ref{sfig:CircNonCollapsed} is defined as:

\begin{equation}
\VorticityForm(v) = \sum_{i=1}^6 *_1(e_i) \cdot \VelocityForm(e_i)
\end{equation}

On the contracted domain $K'$ we have:

\begin{equation}
\VorticityForm'(v') = \sum_{i=2}^5 *_1'(e_i') \cdot \VelocityForm'(e_i') + *_1'(e_{1,6}') \cdot \VelocityForm'(e_{1,6}')
\end{equation}

As edge $e_1$ and $e_6$ are merged to $e_{6,1}'$, we get the equality $\VelocityForm(e_1) = \VelocityForm(e_6) = \VelocityForm'(e_{1,6}')$.
Moreover, we have $\VelocityForm'(e_i') = \VelocityForm(e_i)$ and define $*_1'(e_i') = *_1(e_i)$ for the remaining edges.%$i=2,3,4,5$.
The vorticities defined on both vertices have to be equal, which implies:
%As both vorticities have to be equal in order to get compatible curl operators we get:

\begin{equation}
\label{eq:SimpleHodgeEntryDef}
*_1'(e_{1,6}') = *_1(e_1) + *_1(e_6)
\end{equation}

We take \myeqref{eq:SimpleHodgeEntryDef} as definition for our hodge-star. 
Thus, our formal definition of the hodge-star operator is:
\begin{equation}
(*_1')(e') := \sum_{e \in F_E^{-1}(e')} (*_1)(e)
\end{equation}

\begin{figure}[h!]
	\begin{minipage}[b]{0.5 \linewidth}
		\centering
		\subfigure[]{
					\label{sfig:CircNonCollapsed}
			      	\input{pics/tikz/CircNonCollapsed}}
	\end{minipage}
	\begin{minipage}[b]{0.5 \linewidth}
		\centering
		\subfigure[]{
					\label{sfig:CircCollapsed}
			      	\input{pics/tikz/CircCollapsed}}
	\end{minipage}		
	\caption[Circulation around a vertex $v$ before and after merging]
			{The circulation around vertex $v$ in (a) is defined in terms of its incident edges
			 $e_1 , ... , e_6$. (b) is constructed by merging the edges $e_1$ and $e_6$.}
	\label{fig:VorticityCollidingEdges}
\end{figure}

\newpage
\myparagraph{Proof of correctness}
%It remains to prove, that our method is correct. 
The resulting velocity field has to be divergence free, needs to have the prescribed circulations around holes and the desired inner vorticities.

\paragraph*{}
Take an arbitrary vorticity form $\VorticityForm \in \DiffForms^D_2(K)$ and set the boundary vertices to zero.
We map $\VorticityForm$ to the complex $K'$ and denote it as:

\begin{equation}
\VorticityForm' := \Pi_\VorticityForm(\VorticityForm)
\end{equation}

As $K'$ has a contractible topology, we can derive a divergence free velocity form $\VelocityForm' \in \DiffForms^P_1(K')$ for given boundary fluxes
which satisfies $\DECCurl'\VelocityForm' = \VorticityForm'$.
We map $\VelocityForm'$ back onto the complex $K$ with the velocity transfer operator and denote it as $\VelocityForm$.
$\VelocityForm$ is divergence free since $\VelocityForm'$ is divergence free.
The hodge-star operator has been designed to make the curl operators defined in $K$ and $K'$ compatible for inner vorticities.
Hence,  $\DECCurl \VelocityForm = \VorticityForm$ is true for inner vertices.
It remains to show, that the circulations of $\VelocityForm$ around holes are as desired.
The boundary vertices of a single hole are mapped to some vertex $v' \in V'$.
The vorticity at $v'$ w.r.t. $\VelocityForm'$ is equivalent to the circulation of $\VelocityForm$ around the corresponding hole. 

\eproof

\HeaderThree{Smooth hodge-star on edges}

The main emphasis in this subsection is on arbitrary hodge-stars and in particular the smooth
hodge-star on edges.
An arbitrary hodge-star is a spd matrix $S \in \R^{n \times n}$, where $n$ is the number
of primal and dual edges.
In this subsection, we identify edges by their respective indices $i \in \{ 1 , ... , n \}$.
$S_i$ denotes the i-th row vector and $S_{ij}$ an entry in the j-th column of the i-th row.
Based on a divergence free velocity form $\VelocityForm \in \DiffForms^P_1(K)$, the circulation along the i-th dual edge is defined as:

\begin{equation}
\sum_{j=1}^n S_{ij} \cdot \VelocityForm_j = S_i \VelocityForm
\end{equation}

The matrix $S_i$ maps the 1P-form $\VelocityForm$ to the circulation along the i-th dual edge.
For an inner vertex $v \in V$ let $i_1, ... , i_m$ denote the incident edges and
$\alpha_1 , ... , \alpha_m \in \{-1,+1\}$ their relative orientation to $v$.
The vorticity at $v$ is defined in terms of $\alpha_i$ and $S_i$ as:

\begin{equation}
\sum_{j=1}^m \alpha_j \cdot S_{i_j} \cdot \VelocityForm
\end{equation}

As $v$ is an inner vertex, all the edges $i_1,...,i_m$ are not contracted.
Therefore, $F_E$ maps the edges $i_1,...,i_m$ to existing edges in $K'$.
We identify the edges in $K'$ also by their indices.
First, we try to translate $S_i$ to $K'$ and assume that the edges $i_j$ do not merge.
We denote the hodge-star in $K'$ as $S'$ and define $k := F_E(i)$.
$S_k'$ identifies the k-th row vector of $S'$.
For all $j = 1,...,n$ where $F_E(j) =: l \neq 0$, $S_{kl}'$ is defined as
$O_E(j) O_E(l) S_{ij}$. The remaining entries of $S_k'$ are set to zero.

\paragraph*{}
We defined $S_k'$ based on the edge $i$ in $K$. We denote this as $S_k(i)'$.
In order to handle merging edges, we extend the definition to:

\begin{equation}
S_k' := \sum_{i \in F^{-1}_E(k)} S_k'(i)
\end{equation}

Every primal edge $i$ in $K$ introduces a dual edge whose corresponding circulations is defined as $S_i U$.
$S_k(i)' \Pi_\VelocityForm(\VelocityForm) = S_k(i)' U'$ represents the same circulation in $K'$.
Suppose two different edges $i_1 \neq i_2$ in $K$ get merged to an edge $k$ in $K'$.
Again, $S_k(i_1)' \VelocityForm'$ and $S_k(i_2)' \VelocityForm'$ represent the circulations along their respective dual edges in $K'$.
The dual edges of $i_1$ and $i_2$ merge to the dual edge of $k$.
However, we want that the dual edge of $k$ represents the circulations of the dual edges of its corresponding primal edges in $K$.
Therefore, we add $S_k'(i_1) \VelocityForm'$ and $S_k'(i_2) \VelocityForm'$.
Hence, the associated dual edge of $k$ adds the same circulation as the ones of $i_1$ and $i_2$ in $K$.

\paragraph*{}
We recover the velocity form on the contracted complex $K'$ for given vorticity.
Moreover, we defined the hodge-star for edges on $K'$ to ensure the compatibility with the initial complex $K$.