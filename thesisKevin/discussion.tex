\chapter{Discussion}
\label{ch:discussion}
In this thesis, we present a unified system to handle collisions between rigid and deformable bodies with friction. We modified the integration scheme by Bridson et al. \cite{BRIDSON2002} to handle rigid and deformable bodies, applied a CCD \cite{PROVOT1997} and handled the detected collisions with CPF \cite{TANG2012}. Friction is applied with a penalty based approach \cite{YAMANE2006} providing static and dynamic friction and a straightforward integration in the integration scheme.

We compared the CPF algorithm by Tang et al. \cite{TANG2012} to a DPF algorithm and showed that the CPF provide more stability, less jitter and smaller penetration depths with a small computational overhead if a CCD algorithm is applied for both. DPF in comparison to CPF provide only a notable computational cost advantage if DCD is used, since CCD is more expensive than DCD. % The computational cost for the collision handling at all was small in comparison to the cost of the collision detection.

%We revealed the issues of implausible feature pairs, actually resolved collisions remaining and discretization contact forces for the CPF algorithm.
%Resulting in ghost forces and implausible tangential forces for sliding contacts.

We develop an advanced collision resolution condition and an algorithm based on the redefinition of feature pairs for non-deep intersections to handle the observed artifacts of separated features still assumed colliding, additional artificial collisions through discretization and inconsistent feature pair mapping. Our new algorithm detects artifact collisions, by defining normal cone alike spans for the colliding features and checking their position. The detected artifact collisions are redefined in order to provide a consistent collision handling.

As shown in our benchmarks the new algorithm solves the issue of collisions remaining, reduces the artifact tangential forces induced by the CPF algorithm and provides a more plausible behavior. The algorithm is limited since we keep the CPF normal definition, and thereby in many cases the tangential component can only be reduced.

In the future we want to solve the current drawback of the redefinition algorithm that we can only handle non-deep contacts, since we cannot differentiate between deep contacts and sliding contacts. A possible solution for deep collisions could be to check the neighbors of the new colliding vertex for collisions and to take their contact normal as a reference.  Furthermore it may be interesting to investigate alternative contact normal definitions, taking the adjacent elements of the vertices and edges into account and integrating them into the CPF algorithm.
Moreover, alternative integration of the impulses in the deformable bodies could reduce instability and the inverting of tetrahedrons. A possible approach would be to divide the impulse on the neighboring vertices. Furthermore, the stability could be increased by the application of an implicit integration method. Additionally, an automatic computation of the penalty coefficient would increase the user-friendliness. A possible basis could be the stability ratio of section \ref{sec::stability}: $\Delta t \sim \sqrt{\frac{m}{k}}$. In particular, for scenes with large amounts of colliding features a GPU-based implementation would provide a speedup, since the CPF impulses and the contact categorizations could be computed parallel.