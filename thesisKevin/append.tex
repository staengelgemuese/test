\begin{appendix}
\label{ch:appendix}

\chapter{Discrete Poincaré lemma for 1P-forms in two dimensions}
\label{ap:PoinCLemma}
The Poincaré lemma states that if X is a contractible open subset of $\R^n$ every smooth closed p-form $\alpha$ defined on X is exact.
A discussion of a discrete analog for star shaped simplicial complexes is discussed in \cite{DES01}.
However, we will prove the discrete analog in two dimensions for 1-forms.
Contractible means there are no holes in the complex.
A contractible domain in this context means, that the domain does not contain holes.

\section{Homology group $\mathcal{H}_1$}
\label{sec:HomologyGroup}
In algebraic topology, homology groups characterize the homeomorphic invariant structure of a topological space.
These structural properties are intuitively known as number of connected components, tunnels, holes, voids, etc. .
The homology group $\mathcal{H}_1$ is formally defined as:

\begin{equation} 
\mathcal{H}_1 := \ker \partial_1 / \im \partial_2
\end{equation}

In two dimensions, $\dim \mathcal{H}_1$ equals the number of holes. 
%It characterizes  the number of holes. In fact the dimension of $\mathcal{H}_1$ equals the number of holes. 
$\ker \partial_1$ is the set of all edge-based loops and their linear combinations. Two loops are considered to be equal if their difference is the boundary of a triangle-based chain, i.e., an element of $\im \partial_2$.
In a contractible complex every loop is the boundary of a triangle-chain and thus equal to $0$, as one can see in Figure \ref{fig:BoundaryOperatorNoVoids}. 
%Otherwise in Figure \ref{fig:BoundaryOperatorVoids} is illustrated that in a complex with holes there are loops which are not equal to $0$.
In contrast, complexes with holes contain loops which are not equivalent with $0$ (see Figure \ref{fig:BoundaryOperatorVoids}).  
In our case the dimension of $\mathcal{H}_1$ is $0$ and thus $\ker \partial_1 = \im \partial_2$.

\begin{figure}[htbp]
	\centering
 	\input{pics/tikz/nonVoidBoundary}
	\caption{The boundary operator is applied to a spherical chain which doesn't contain holes. Therefore, the boundary is a simple loop.}
	\label{fig:BoundaryOperatorNoVoids}
\end{figure}	

\newpage

\begin{figure}[htbp]
	\centering
 	\input{pics/tikz/VoidBoundary}
	\caption{The boundary operator is applied to a chain which contains a hole. A simple loop around the hole cannot be the boundary of some chain as the boundary of the hole is always included.}
	\label{fig:BoundaryOperatorVoids}
\end{figure}	

\section{Proof}
\begin{theorem}
$\ker d_2$ is the set of closed 1-forms and $\im d_1$ is the set of the exact forms. 
So $\ker d_2 \subset \im d_1$ iff every closed 1-form is exact.
From $d_2 \circ d_1 = 0$ follows $\im d_1 \subset \ker d_2$. It suffices now to prove $\dim \ker d_2 = \dim \im d_1$.
Since $d_i = \partial_i^T$ and $\rnk \partial_i = \rnk \partial_i^T$ it follows $\dim \im d_i = \dim \im \partial_i$.
The final conclusion is:

\begin{eqnarray*}
\dim \ker d_2 = \dim \Omega_d^1 - \rnk d_2 = \dim \ker \partial_1 + \rnk \partial_1 - \rnk \partial_2 =^* \rnk \partial_1
\end{eqnarray*}

* follows from $\ker \partial_1 = \im \partial_2$ which has been motivated in the previous Section \ref{sec:HomologyGroup}.
\eproof
\end{theorem}

\end{appendix}
