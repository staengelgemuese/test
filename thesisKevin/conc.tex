\chapter{Conclusion}
\label{ch:Conclusion}

In this thesis we applied the approach of \myciteS{elcott} to regular grids in order to avoid computational intensive mesh constructions.
For this reason, we translated the required concepts of DEC onto regular grids.
We applied bilinear interpolation to regular grids to reconstruct the velocity field.
Additionally, we introduced a dense hodge star to make the interpolation compatible with DEC.
Furthermore, we extended the advection step through adaptive sub-sampling and applied a MacCormack method to increase the stability.
Due to a dense hodge star operator, our approach has comparable vorticity conservation properties.
The implementation of boundary fluxes has been analyzed, e.g., for simulating a wind tunnel.
Moreover, to simulate obstacles inside the domain we studied non-trivial topologies for regular grids and simplicial complexes.
%Boundary fluxes and obstacles inside the domain have been extensively studied and implemented.
We analyzed the pressure distributions of airfoils within a free stream and compared the pressure coefficients to existing methods.%with JavaFoil.
With sufficient spatial discretization the pressure distributions of our method are comparable to literature.

The major disadvantage of regular grids is the incapability of representing accurately curved boundaries as we have seen in Subsection \ref{sec:ResultPressure}.
For future work, a combination of regular grids and simplicial complexes should be investigated in order to model curved surfaces without complicated mesh creations.
Paralleled implementations of the regular grid approach may be interesting as they are well suited for parallel operations and thus promises substantially performance increases.
%Paralleled implementations of the regular grid approach may be interesting as GPU based implementations promises substantially performance increases.
Finally, higher order interpolations with corresponding smoother hodge-stars, e.g., bicubic spline interpolation, may even more reduce the vorticity dissipation.

\label{ch:conc}