\chapter{Discrete exterior calculus on regular grids}
\label{ch:grids}

As we have seen in Section \ref{sec:SimplicialComplex}, the original version of DEC uses simplicial complexes to discretize the simulation domain. 
The quality of the dual mesh is closely related to the \newword{well-centeredness} of the primal mesh.
A triangle is called \newword{well-centered} if its circumcenter is located inside the triangle.
\newword{Well-centeredness} as a global mesh property is defined by \myciteS{HI01}.
The property measures in a continuous fashion the quality of the mesh w.r.t. how many triangles are well-centered.

Unfortunately, it is very complicated to create a good well-centered mesh.
Therefore, we will present an alternative discretization for DEC in this chapter based on \newword{regular grids}.
As in the previous chapters, we will again restrict to the two dimensional case.
Regular grids divide the domain in equally shaped rectangles. 
Figure \ref{fig:RegGrid} shows an example of a regular grid containing a hole in its center.

\begin{figure}[htbp]
	\centering
 	\input{pics/tikz/regGrid}
	\caption{A regular grid containing a hole in its center.}
	\label{fig:RegGrid}
\end{figure}	

A rectangle is trivially well-centered. Hence, the dual mesh of a regular grid is always well shaped.
Another advantage of the discretization is the regularity, which enables fast structural operations, e.g., locating a position or tracing particles.
The formal definition of a regular grid is as follows:

\begin{definition}[Regular grid]
A \newdef{regular grid} of size $n,m \in \N$ is a tuple of the  following form:

\begin{equation}
R:=(n,m,g)
\end{equation}

where $g$ is defined as:

\begin{equation}
g : \{ 0 , ... , n - 1 \} \times\{ 0 , ... , m - 1 \} \rightarrow \{0,1\}
\end{equation}

\end{definition}

The tuple $(n,m,g)$ defines a regular grid with $n \times m$ cells.
$g$ removes cells by flagging them with $0's$ in order to enable non-trivial domains, i.e., domains with obstacles.

\HeaderTwo{Chains and forms}

In order to define DEC on regular grids, we first have to define chains and forms.
The discrete forms are the cochains, i.e., linear functions which assign real numbers to chains.
Therefore, it is essential and sufficient to define chains on regular grids.

\myparagraph{Chains}
A chain is defined as a weighted sum of simplices of a specific dimension (see Section \ref{sec:SimplicialComplex}).
However, regular grids are not composed of simplices, as they consists of rectangles instead of triangles.
On regular grids, we solve this, by defining 2-chains to be weighted sums of rectangles.

\HeaderTwo{Discrete Exterior derivative}

The discrete exterior derivative is just the adjoint of the boundary operator as we have seen in Section \ref{sec:DECDerivative}.
Therefore, we have to setup the boundary operators for edges and faces in order to define the exterior derivative.
Now, vertices, edges and faces (rectangles) are denoted by $V$, $E$ and $F$ respectively.

\myparagraph{Edges}
The boundary operator for edges can be represented as a matrix $\partial_1 \in \R^{|V|\times|E|}$.
An edge $e \in E$ points from some start vertex $v_0 \in V$ to an end vertex $v_1 \in V$.
The associated column in $\partial_1$ contains a $-1$ for $v_0$ and $1$ for $v_1$.
The discrete exterior derivative for vertices is then defined as $\partial_1^T$.

\myparagraph{Faces}
Faces are assumed to be counter-clockwise (CCW) oriented and induce an orientation onto their boundary edges, as illustrated in Figure \ref{fig:GridInducedOrientation}.
In a regular grid every edge is either parallel to the X- or Y-axis and is assumed to be oriented into the positive direction of its corresponding axis (see Figure \ref{fig:GlobalEdgeOrientation}).

\begin{figure}[h!]
	\begin{minipage}[b]{0.5 \linewidth}
		\centering
		\subfigure[The orientation of primal edges induced through the primal faces.]{
			      \input{pics/tikz/gridinducedorientation}
			      \label{fig:GridInducedOrientation}}
	\end{minipage}
	\begin{minipage}[b]{0.5 \linewidth}
		\centering
		\subfigure[The global orientation of the primal edges.]{
			      \input{pics/tikz/gridglobalorientation}
			      \label{fig:GlobalEdgeOrientation}}
	\end{minipage}	
\caption[Induced and global orientation of primal edges.]{}
\end{figure}

For a face $f \in F$, the corresponding column vector $c \in \R^{|E|}$ of the boundary matrix $\partial_2 \in \R^{|E|\times|F|}$ flags
boundary edges with $1$ or $-1$ if the induced orientation matches with the global.

\newpage
\HeaderTwo{Deriving vector fields from 1-forms on regular grids}
In Subsection \ref{ssec:TimeIntegration}, we constructed a velocity field from its associated form based representation.% in order to perform particle tracing.
We will do this on regular grids by computing a vector for each primal face (dual vertex) and apply a bilinear interpolation to each dual cell.

\begin{figure}[h!]
	\centering
 	\input{pics/tikz/regGridInt}
	%\caption{The velocity vector $\mathbf{u}$ located at the center is reconstructed from the given boundary fluxes $U_1^x$, $U_2^x$, $U_1^y$ and $U_2^y$.}
	\caption[Velocity field reconstruction]{The velocity vector located at the center is reconstructed from the given boundary fluxes.}
	\label{fig:RegGridInt}
\end{figure}

Figure \ref{fig:RegGridInt} illustrates the reconstruction of a vector from a 1-form with values $U_1^x$, $U_2^x$, $U_1^y$ and $U_2^y$ that are defined on the boundary edges. 
The velocity vector $\mathbf{u}$ located at the center is defined as:

\begin{equation}
\label{eq:RedGridVelVector}
u_x := - \frac{U_1^x + U_2^x}{ 2 }  \text{ , } u_y := \frac{U_1^y + U_2^y}{ 2 }
\end{equation}

\newpage
\HeaderTwo{Dual mesh}

%The dual mesh is the discrete counterpart of the metric and is connected to the primal mesh through the discrete hodge star.
We defined the dual mesh to be the Voronoi diagram of the underlying vertices in Section \ref{sec:DualComplex}.
Obviously, the Voronoi diagram of a regular grid is again a regular grid, which is just the shifted original grid as depicted in Figure \ref{fig:RegGridDual}.

\begin{figure}[htbp]
	\centering
 	\input{pics/tikz/regGridDual}
	\caption{The dual mesh is the shifted primal regular grid.}
	\label{fig:RegGridDual}
\end{figure}

\HeaderThree{Compatibility of hodge star and interpolation}
\label{ssec:ModifiedHodge}
In Subsection \ref{ssec:TimeIntegration} we advected the vorticity along the velocity field to derive the vorticity of the next step.
To apply Kelvins Circulation theorem, linear interpolation has been used in order to evaluate the circulation along the boundaries of the dual faces.
However, interpolation is not a concept of DEC. Therefore, it is not self-evident that our interpolation on regular grids is compatible with DEC operators.
Let us denote the advection process as a function of the following form:

\begin{equation}
\adv : \R^+ \times \{ \VelocityForm \in \DiffForms^P_1(R) : d \VelocityForm = 0 \} \rightarrow \DiffForms^D_2(R), 
(\Delta t,\VelocityForm) \mapsto \adv(\Delta t,\VelocityForm)
\end{equation}

$\adv$ takes a divergence free velocity form $\VelocityForm$ as input and advects its vorticity $\Delta t$ seconds along the velocity field.
The operator implicitly computes the vorticity for a given velocity form based on Stokes theorem.
We can now define the curl operator as the limit of the $\adv$ function as $\Delta t$ goes to $0$.

\begin{definition}
The \newdef{interpolation curl} operator is defined as:

\begin{equation}
C_I :  \{ \VelocityForm \in \Omega^P_1(R) : d \VelocityForm = 0 \} \rightarrow \DiffForms^D_2(R) , C_I(\VelocityForm) := \lim_{\Delta t \rightarrow 0} \adv(\Delta t,\VelocityForm)
\end{equation}

\end{definition}

As our new operator derives vorticity, it should be equivalent with the DEC version defined in Section \ref{sec:VortToVel}.
To compute the vorticity of a dual face, both operators sum up the oriented circulations along the dual boundary edges.
To evaluate the circulation along a dual edge, DEC uses the hodge star. 
$C_I$ on the other hand, evaluates the velocity field at the dual vertices and integrates linearly on each dual edge.

\paragraph*{}
We denote the associated velocity vectors of the dual vertices $A$ and $B$ from Figure \ref{fig:IntCirculation} as $\mathbf{A^u}$ and $\mathbf{B^u}$, respectively.

\begin{figure}[h!]
	\centering
 	\input{pics/tikz/intCirculation}
	\caption[Circulation along a dual edge]{$A$ and $B$ are the vertices of the blue dual edge in the middle. The fluxes through the primal edges are identified by $U^x_i$.}
	\label{fig:IntCirculation}
\end{figure}

The circulation along the dual edge is then defined as:

\begin{equation}
c := ( B - A )^T \cdot \frac{\mathbf{A^u} + \mathbf{B^u}}{2}
\end{equation}

The edge is parallel to the X-axis and hence the equation can be simplified to:

\begin{equation}
c = ( B_x - A_x ) \cdot \frac{A^u_x + B^u_x}{2}
\end{equation}

As $B_x - A_x$ is $1$, $A^u_x$ and $B^u_x$ can be defined in terms of the fluxes $U_i^x$:

\begin{eqnarray}
A^u_x = \frac{U_0^x + U_1^x}{2} \\
B^u_x = \frac{U_1^x + U_2^x}{2}
\end{eqnarray}

Finally, the circulation along the dual edge is:

\begin{equation}
\label{eq:IntHodge}
c = \frac{1}{4}( U_0^x + 2 \cdot U_1^x + U_2^x )
\end{equation}

Based on the hodge star $*_1$, the circulation is defined as $c := U_1^x $.
Hence, the interpolation is not directly compatible with DEC.

\paragraph*{}
We take \myeqref{eq:IntHodge} as definition for a smoother hodge-start in order to solve the compatibility issue.
%A more detailed discussion of the smooth hodge-star is given in Appendix \ref{ap:SPDHodgeStar}.








\section{Smooth hodge star}
\label{sec:SPDHodgeStar}
In this section we consider a regular grid $R:=(n,m,g)$ and redefine the hodge star $*_1$ in order to make it compatible with the bilinear interpolation.
We refer to the modified version as \newword{smooth hodge star} that is derived from the advection operator.
As a consequence, the hodge star becomes smoother because it takes also the neighboring edges into account.

\begin{definition}[Component]
A collection of cells in $R$ is called \newdef{component} if the cells satisfy the following conditions:
\begin{itemize}
\item All cells are located in the same row or column.
\item The cells are all connected, i.e., they are not separated through a hole.
\item It is not possible to add a new cell s.t. the collection still satisfies the previous two conditions.
\end{itemize}
\end{definition}

The components are just the rows and columns in a rectangular shaped domain without holes.
The smooth hodge star is closed in a component w.r.t. its transversely aligned primal edges.
First, we will discuss the new hodge star in a single component and later extend it to the whole complex.

\paragraph*{}
To simplify matters, we consider a component of just three cells aligned parallel to the X-axis as illustrated in the following figure:

\begin{figure}[htbp]
	\centering
 	\input{pics/tikz/regGridComponent}
	\caption[A component of three cells.]{A component of three cells. Components are the largest connected parts in a row or column.}
	\label{fig:RegGridComponent}
\end{figure}	

The associated smooth hodge star has the following form:
\begin{equation}
\frac{1}{4} \cdot
\begin{pmatrix} 2 & 1 & 0 & 0 \\ 1 & 2 & 1 & 0 \\ 0 & 1 & 2 & 1 \\ 0 & 0 & 1 & 2  \end{pmatrix} 
\end{equation}

The generalized shape defined on a component of the smooth hodge star is:
\begin{equation}
\frac{1}{4} \cdot
\begin{pmatrix} 2 & 1 & & & \\ 1 & 2 & 1 & & \\  & \ddots & \ddots & \ddots & \\  &  & 1 & 2 & 1 \\ & & & 2 & 1  \end{pmatrix} 
\end{equation}

A vertically oriented component leads to an equivalent matrix.
The determinant of a component of size $n$ is $\frac{1}{4} \cdot (n+1)$.
Following Sylvester’s criterion(\myciteS{G01}), the smooth hodge-star defined on a component is a symmetric and positive definite matrix (spd).

\paragraph*{}
Now, we extend the smooth hodge star from a single component to the whole complex.
Let $m \in \N$ be the number of primal or dual edges.
Every primal edge is located in exactly one component. 
Therefore, we can choose a permutation matrix $P \in \R^{ m \times m }$ s.t. all transversely aligned edges within a component are successively ordered. 
Thus, the complete smooth hodge star ${S_1}'$ with the induced order from $P$ is a block diagonal matrix.
Each block corresponds to a smooth hodge star defined on a single component. Hence, the whole matrix ${S_1}'$ is spd.
Based on the original order of edges, the smooth hodge star is defined as $S_1 := P^T \cdot {S_1}' \cdot P$.
Obviously, $S_1$ is also spd.

\paragraph*{}
We carefully translated the necessary concepts of DEC to regular grids.
Furthermore, we introduced a smoother hodge-star for edges to ensure vorticity conservation, i.e., make the interpolation compatible with DEC.