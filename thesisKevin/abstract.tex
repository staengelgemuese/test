\begin{abstract}
%The development of efficient and stable fluid simulations is a challenging task in computer graphics.
%\myciteS{elcott} describe an approach, based on Discrete Exterior Calculus for simulating the fluid flow.
%A vorticity based formulation of the incompressible Navier-Stokes equations is used, resulting in a mass-conserving representation of the velocity field by definition.
%This approach preserves vorticity at a discrete level, resulting in a visually more realistic fluid flow.
%
%We extend this approach to regular grids in two dimensions.
%So, we avoid computationally expensive mesh constructions.
%We discuss non-trivial boundary conditions and arbitrary topologies.
%The vorticity conservation properties are compared with the classical mesh based approach of Elcott et. al.
%We especially analyze the corresponding pressure fields near the boundaries of inner objects.

Collision handling has been an active research topic in the area of the physically-based simulation of rigid and deformable bodies for many years. A common approach in interactive environments are discrete penalty forces, computing a repulsion forced based on the penetration at one moment in the time step. They provide low computational costs and good scalability, though they suffer from jitter and instability. Tang et al. \cite{TANG2012} improved the approach of discrete penalty forces and introduced 2012 the continuous penalty forces (continuous penalty forces), continuously accumulating penalty forces along the penetration trajectory over the whole time step. Thereby, the jitter and instability issues are reduced. Although, the continuous penalty forces show artifacts especially for enduring contacts, precluding the simulation of sliding contacts.

In this thesis, we present a unified system to handle collisions between rigid and deformable bodies with friction. We modify the integration scheme by Bridson et al. \cite{BRIDSON2002} to handle rigid and deformable bodies, apply a CCD \cite{PROVOT1997} and handle the detected collisions with continuous penalty forces \cite{TANG2012}.
We discuss the artifacts arising from the continuous penalty forces algorithm, examine methods to tackle them and apply the new methods to the continuous penalty forces algorithm. Finally, we analyze the results of the continuous penalty forces algorithm in comparison to discrete penalty forces, evaluate our new algorithm to handle the continuous penalty forces artifacts and inspect further improvements.


%Continuous penalty forces (continuous penalty forces) \cite{TANG2012} improve this approach and compute collision impulses by continuously accumulating penalty forces along the penetration trajectory over the whole time step and thereby reducing the jitter and instability issues.

%A drawback of the continuous penalty forces algorithm is that it shows implausible collision pairs for resting and sliding contacts and implausible contact normals withdrawing an visually satisfying result. 
%Therefore, it is desirable to enhance continuous penalty forces to handle these situations properly. There is already an extensive amount of work about consistent penetration depth estimation and contact normal computation, nevertheless the continuous penalty forces algorithm requires its own contact normal definition. Therefore, we want to apply an algorithm keeping the continuous penalty forces contact normal definition and providing plausible collision feature couples.

%This thesis focuses on analyzing and comparing the continuous penalty forces \cite{TANG2012} and discrete penalty forces when resolving collisions between rigid and deformable bodies in interactive multi body frameworks.
%Furthermore a penalty based friction modeling \cite{YAMANE2006} will be investigated.

%Our investigations show implausible collision pairs for resting sliding contacts and implausible contact normals as well as a condition implying an insufficient collision resolution condition arising from the continuous penalty forces algorithm. We analyze these artifacts and examine several approaches to tackle them.

%An example for a constraint-based approach is the use of discrete penalty forces (discrete penalty forces). The forces only depend on the penetration depth and the contact normal at one moment in the time step of the simulation. Therefore, they provide low computational costs and good scalability. In its basic formulation, this approach suffers from several problems. In particular, discrete penalty forces yield discontinuous forces implying jitter and instability, global inconsistency and interpenetrations.

\end{abstract}