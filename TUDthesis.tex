\documentclass[article,dr=phil,type=drfinal,colorback,accentcolor=tud9c]{tudthesis}
\usepackage{ngerman}


\begin{document}
 
 
  \section{Generelle Informationen}
    Die Klasse basiert auf der \textaccent{tudreport}-Klasse von C. v. Loewenich und
    J. Werner. Alle "Anderungen dort wirken sich direkt auf die
    \textaccent{tudthesis}-Klasse aus. Genauer: die \textaccent{tudthesis}-Klasse definiert nur einige
    neue Befehle und legt die Formatierung der ersten zwei Seiten (Titelseite
    und R"uckseite des Titleblattes) fest. \textbf{Alle Vordefinierten Texte sind, wie verbindlich vorgeschrieben, in der hessischen Amtssprache
    gehalten\footnote{Deutschland hat (noch) keine Amtssprache.}.}

  \section{Verwendung der Klasse}
    Die Klasse wird verwendet, indem in der Dokumentenpr"aambel
    \textaccent{\textbackslash documentclass\{tudthesis\}}
    eingetragen wird.

 


\end{document}
