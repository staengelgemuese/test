\documentclass[article,dr=phil,type=drfinal,colorback,accentcolor=tud2a]{tudthesis}
%\usepackage{ngerman}

\newcommand{\getmydate}{%
  \ifcase\month%
    \or Januar\or Februar\or M\"arz%
    \or April\or Mai\or Juni\or Juli%
    \or August\or September\or Oktober%
    \or November\or Dezember%
  \fi\ \number\year%
}

\begin{document}
  \thesistitle{Numerical simulation of compressible flows with immersed boundaries using Discontinuous Galerkin methods}
    {The Fabulous Benner Boys present lordly the {\LaTeX} TUD corporate design!}
  \author{Simone Katharina Stange}
  \birthplace{Mainz}
  \referee{Clemens v. Loewenich}{Johannes Werner}[another one]
  \department{Fachbereich Physik}
  \group{Fachgebiet Fluiddynamik}
  \dateofexam{\today}{\today}
  \tuprints{12345}{1234}
  \makethesistitle
  \affidavit{S. Stange}
  
  \tableofcontents
  \newpage
  \listoffigures
  \newpage
	
  \section{Abstract}
  \section{Motivation}
  \section{Discontinous Galerkin Method}
  \section{Immersed Boundaries}
    \subsection{Cell Agglomeration}
  \section{BoSSS}
  \section{Generelle Informationen}
    Die Klasse basiert auf der \textaccent{tudreport}-Klasse von C. v. Loewenich und
    J. Werner. Alle "Anderungen dort wirken sich direkt auf die
    \textaccent{tudthesis}-Klasse aus. Genauer: die \textaccent{tudthesis}-Klasse definiert nur einige
    neue Befehle und legt die Formatierung der ersten zwei Seiten (Titelseite
    und R"uckseite des Titleblattes) fest. \textbf{Alle Vordefinierten Texte sind, wie verbindlich vorgeschrieben, in der hessischen Amtssprache
    gehalten\footnote{Deutschland hat (noch) keine Amtssprache.}.}
  \section{Conclusion}
  

\end{document}
